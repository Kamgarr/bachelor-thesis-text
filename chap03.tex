\chapter{Review of related software libraries}
\label{chapt:technology}
In this chapter, we will define the requirements for running our application and introduce used libraries.

All selected libraries support running on both \emph{Linux} and \emph{Windows} systems. This demo was implemented and tested on \emph{Ubuntu 17.04} with packages described in this chapter. Code of our application is entirely written in \emph{C++} using C++14 standard, to compile the code we used \emph{g++} version 4:6.3.0.

To properly run our application requires a computer with a processor having at least four threads and 8GB of RAM.

\section{Physics engine}
Bullet Physics


\section{Graphics engine}
Irrlicht Engine

\section{Geometric library}
CGAL

\section{Generation of Voronoi cells}
Voro++ is a library for carrying out three-dimensional computations of the Voronoi tessellation. It calculates Voronoi cell for each particle individually and is suited for high performance calculations on large scale particle systems. It is also able to clip Voronoi cells to any user defined boundary.

This library would be well suited for decomposing whole objects into Voronoi cells and has a very minimalistic interface. Because implementation requires only one Voronoi cell per collision, this library can provide a simple and fast solution for our task.

\section{Library for convex decomposition}
\label{sec:decompositionLib}
HACD (Hierarchical Approximate Convex Decomposition
\cite{HACD}





