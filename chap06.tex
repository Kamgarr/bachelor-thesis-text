\chapter{Installation guide}
%\addcontentsline{toc}{chapter}{Appendix - Installation guide}
\addtocontents{toc}{\protect\setcounter{tocdepth}{0}}

This section documents the process of compiling and running the application. We will focus on the installation process on \emph{Ubuntu 17.04}, but the code is written in as platform independent and should work on any platform where the library dependencies work.

The packages with library dependencies are available for multiple \emph{Linux}
distributions and can be manually compiled and installed on most other
operating systems.

\paragraph{Dependencies}
Following packages are required to compile the application. We also provide versions of packages used to create and test our implementation.
\begin{center}
\begin{tabular}{ll}
package & version \\
\hline
libbullet-dev & 2.83.7+dfsg-5 \\
libirrlicht-dev & 1.8.4+dfsg1-1 \\
libcgal-dev & 4.9-1build2 \\
voro++-dev & 0.4.6+dfsg1-2 \\
HACD & bundled with the application\footnote{downloaded from \url{https://sourceforge.net/projects/hacd}}
\end{tabular}
\end{center}
\todo{notice: ctverecky kolem tabulek jsou fuj.}

A C++ compiler capable of compiling the C++14 standard is also needed, we have used GCC version \todo{doplnit}XXX with flags YYY.

\section{Compiling and running the application}
We assume that all the right versions of the packages from the previous section are correctly installed. The HADC library is included with our code with its own {\tt Makefile}.

\subsection*{Using make}
The bundled HACD library needs to be compiled first, then the main application is compiled. The compiling process may take a few minutes.\todo{poznamku o -jN jsem smazal protoze oponent prosel zakladni skolou a umi unix.}
\begin{code}
\~/bachelor_thesis\$ make -C lib/hacd/
\~/bachelor_thesis\$ make
\~/bachelor_thesis\$ build/game
\end{code}
\todo{fakt to mas v adresari bachelor\_thesis? Mozna by se hodilo to prejmenovat, treba na DestructionInGame nebo cokoliv podobne generickyho. Pridej tam tar -xzf a cd, a to spusteni hry oddel, protoze s nazvem sekce `using make' uz vubec nesouvisi.}

\subsection*{Manual compilation}
\todo{Je tohle fakt potreba? Existuje nejaka platforma kde zaroven je GCC a neni make? (ne)}
Here we are going to exclusively focus on compiling our own code using a \emph{g++} compiler version 5 or higher, supporting {\tt c++14}. All of our {\tt cpp} and their respective header files can be found in {\tt /src} directory.
Following flags are needed to compile individual modules into object files:
\begin{code}
-std=c++14
-frounding-math
-isystem /usr/include/bullet 
-isystem /usr/include/irrlicht 
-isystem /usr/include/bullet/LinearMath 
-isystem include (HACD header files)
\end{code}

Linking object files to make a final application requires adding the following libraries: (HADC library also need to be linked to our application, link it as a static or dynamic library depending on installation.)
\begin{code}
-lIrrlicht 
-lBulletSoftBody 
-lBulletDynamics 
-lBulletCollision    
-lLinearMath 
-lvoro++ 
-lgmp 
-lCGAL 
-lCGAL_Core 
-lmpfr 
-lpthread
HACD.a or -lHACD (second option may require -L/path/to/library)
\end{code}

\todo{Tady se jeste hodi rozebrat co v tom adresari vlastne je (kde jsou jaky media, modely atd., a ten konfiguracni textovej soubor, a jak se to spousti, jakej hardware to asi zhruba bude potrebovat a na jakym hardwaru jsi to (zevrubne popsano) testoval.}
