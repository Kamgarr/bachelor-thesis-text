\chapter*{Conclusion}
\addcontentsline{toc}{chapter}{Conclusion}

This thesis has reviewed currently used techniques used to simulate destructible environments, implemented a simple game environment to test some of the approaches and designed and measured an approach based on a combination of the reviewed techniques. Thesis brings following results and conclusions:

\begin{itemize}
\item We have successfully presented an approach that combines boolean operations on 3D objects and Voronoi tessellation. The performance experiment~\cref{sec:overallperformance} has shown that the visual experience has not been impacted by presenting destruction multiple frames later as a result of multiple thread solution with a delayed application.

\item The experiments have concluded that used techniques are viable for low polygon meshes. For the boolean operations, we could tolerate meshes with the size of about 2000 triangles.The convex decomposition has reached the critical time at around 1000 triangles. The limits have been measured by the experiment designed in the~\cref{sec:triangleperformance}.

\item We have tested ~\citet{edem} approach in the same setting as our implementation. The result in \cref{sec:edemtest} strongly suggests that this approach is applicable only to objects divided into a small number of elements.
\end{itemize}


\subsection*{Future work}
The implementation showed several points that might be viable as starting points for future research and work.
\begin{itemize}
\item We have used CGAL library to perform boolean operations on 3D meshes, the approach it implements has performed consistently, however, CGAL is a geometric library, and it is designed to provide a rich interface with multiple views on data. For the destructible environment, we propose a library for boolean operations on 3D triangular meshes optimized for use in real-time environments.
\item The implementation has shown the problem of determining a centre of gravity for arbitrary closed 3D triangular objects with a homogeneous distribution of mass, after dynamically creating new meshes.
\item We have not incorporated texture generation into implementation. Instead, we relied on lightning to show the object's features. The generation of the textures for the fragments of the original textured object would benefit this work.
\item Most of the modern game engines do not perform stability analyses on the damaged objects. We have envisioned independent stability analyses running in a separate thread, testing recently damaged objects.
\todo{ja by som kolizie lietadla vobec neriesil, keby to bola normalna hra tak to napise game over a done.}
\end{itemize}

