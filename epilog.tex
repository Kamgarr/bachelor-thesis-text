\chapter*{Conclusion}
\addcontentsline{toc}{chapter}{Conclusion}

This thesis has reviewed currently used techniques used to simulate destructible environments, implemented a simple game environment to test some of the approaches and designed and measured an approach based on a combination of the reviewed techniques. Thesis brings following results and conclusions:

\begin{itemize}
\item We have successfully presented an approach that combines boolean operations on 3D objects and Voronoi tessellation. The performance experiment~\cref{sec:overallperformance} has shown that the visual experience have not been impacted by presenting destruction multiple frames later as a result of multiple thread solution with a delayed application.

\item The experiments have shown the limits of used approaches and libraries. 
For the boolean operations, we could tolerate meshes with the size of about 2000 triangles.The convex decomposition has reached the critical time at around 1000 triangles. The limits have been measured by the experiment designed in the~\cref{sec:triangleperformance}. 
\end{itemize}



Koment: Conclusion je prakticky tosamy jako thesis goal statement v introductionu, akorat je napsany v past perfect a narozdil od referenci ktery ukazujou kde co budes resit tam davas reference ktery odkazujou na zajimavy veci (vysledky reserse a mereni) ktery si pritom zjistil. Ve vysledku muzes naprosto naplno rict co si o tom myslis. Vsechny nevyhody a prusvihy co si zpusobil muzes shrnout do...

\subsection*{Future work}
fix tazisko a spawn point
textury
statika

bullet kolizie lietadla
tady zhruba popises co jsi vynechal a proc (treba ze to bylo moc slozity a out of scope of the thesis) ale asi by si to zaslouzilo na tom zamakat. Zjistili sme ze rozklad typu X je dementni protoze Y, takze priste pouzijem Z. Zjistili sme ze nef polyhedra je slozity vytvaret, takze by bylo dobry mit nejakou knihovnu nebo format dat kterej to umi zvladnout chytrejc. Atd.
