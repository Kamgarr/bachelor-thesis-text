\chapter{Review of related software libraries}
\label{chapt:technology}
In this chapter, we will define the requirements for running our application and introduce used libraries.
\todo {text here}


\section{Physics engine}
Bullet Physics
\todo {text here}


\section{Graphics engine}
Irrlicht Engine
\todo {text here}

\section{Geometric library}
CGAL
\todo {text here}

\section{Generation of Voronoi cells}
Voro++ is a library for carrying out three-dimensional computations of the Voronoi tessellation. It calculates Voronoi cell for each particle individually and is suited for high performance calculations on large scale particle systems. It is also able to clip Voronoi cells to any user defined boundary.

This library would be well suited for decomposing whole objects into Voronoi cells and has a very minimalistic interface. Because implementation requires only one Voronoi cell per collision, this library can provide a simple and fast solution for our task.

\section{Library for convex decomposition}
\label{sec:decompositionLib}
HACD (Hierarchical Approximate Convex Decomposition
\cite{HACD}
\todo {text here}





