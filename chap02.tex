\chapter{Review of related software libraries}
\label{chapt:technology}
In this chapter we introduce libraries that can be beneficial for creating a destructible environment. We do not talk about all-in-one software development kits with their own approach to destructible environment already implemented. Instead our focus is on building a game with destructible environment from zero. 

To build a game we will surely need a physics engine, a graphics engine, and to support our destructible environment a selection of geometric libraries. We mention which libraries we used in the implementation and details of their use can be found in~\cref{app:implementation}.

\section{Physics engine}
Bullet Physics
\todo {text here}


\section{Graphics engine}
Irrlicht Engine
\todo {text here}

\section{Geometric libraries}
CGAL
\todo {text here}

\subsection{Boolean operations}

\subsection{Voronoi tessellation}

\begin{description}
\item[Voro++] is a library for carrying out three-dimensional computations of the Voronoi tessellation. It calculates Voronoi cell for each particle individually and is suited for high performance calculations on large scale particle systems. It is also able to clip Voronoi cells to any user defined boundary.

This library would be well suited for decomposing whole objects into Voronoi cells and has a very minimalistic interface. Because implementation requires only one Voronoi cell per collision, this library can provide a simple and fast solution for our task.
\end{description}


\subsection{Convex decomposition}
\label{sec:decompositionLib}
HACD (Hierarchical Approximate Convex Decomposition
\cite{HACD}
\todo {text here}





