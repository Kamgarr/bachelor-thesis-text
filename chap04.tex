\chapter{User Documentation}
%\addcontentsline{toc}{chapter}{Appendix - Documentation}
\addtocontents{toc}{\protect\setcounter{tocdepth}{0}}
This part of the documentation is focused on the end user, in this case, player. We introduce controls of the application and detail the possibility to adjust the game world.
\section{User interface}
Here, we provide the list of parameters and key bindings needed to control our application.
\subsection*{Parameters}
The application provides only one console parameter. It is {\tt -d}, and it enables debugging information to be rendered. In other words, it draws the collision shapes and bounding boxes around the objects.
\subsection*{Keyboard input}
Keys can be pressed simultaneously and also held down instead of multiple presses. The bindings are:
\begin{itemize}
\item W/S: pitch down/up
\item A/D: roll counter-clockwise/clockwise 
\item Q/E: yaw left/right
\item Z/X: speed up/slow down
\item Space Bar: shoot or unpause
\item P: pause the game
\item ESC: quit the game
\end{itemize}


\section{Input data}
\label{sec:data}
To allow the user to easily change the game world without the need of recompilation the input text file describing the whole setting is provided. The mentioned file is located in {\tt media/world.cfg}. 

\subsection*{{\tt world.cfg} format}
We will show that the structure of the data is straightforward and therefore we have chosen not to use XML or any other standardised format. The file is interpreted by lines. Each line represents one in-game entity. First three lines have special meaning, and the rest are destructible objects. Inside the line, we use semicolons as a separator between data items.

The \emph{first line} describes the sky-box, it is formed of 6 images in given order: top, bottom, left, right, front, back; relative to the starting viewpoint. 

All other lines use same nine data items. Those are object model, texture, position (3 coordinates), scale (3 numbers, one for every axis) and mass. Every number is used as a floating point number.
The second line describes the ground. Currently, the ground is generated as a solid cube with 1x1x1 size. Therefore model file is not needed. The third line is a description of our controlled vehicle. Beginning with this line, the texture file is an optional parameter and can be left blank. Bellow, we can see an example file.

\begin{centering}
\begin{Verbatim}[frame=single,numbers=left,xleftmargin=5mm]
top.jpg;bottom.jpg;left.jpg;right.jpg;front.jpg;back.jpg
;grid.jpg;0;-15;0;1000;10;1000;0
fighter.3ds;;0;20;0;0.5;0.5;0.5;1
building.obj;;-20;-11;-30;2;2;2;500
pyramid.obj;;50;-11;-100;5;10;5;800
...
\end{Verbatim}
\end{centering}
The zero mass means that the object is static and does not move.

\subsection*{Object models and textures}
The input formats are limited to the support of \emph{Irrlicht}. The detailed list of supported object and texture formats can be found on \emph{Irrlicht} web page~\footnote{http://irrlicht.sourceforge.net/?page\_id=45\#supportedformats}
\todo {something about manifolds and compatibility with nef_polyhedron}


